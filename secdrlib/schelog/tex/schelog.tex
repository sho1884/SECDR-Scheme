\documentstyle[]{jarticle}
\setcounter{secnumdepth}{6}
\setcounter{tocdepth}{6}
\topsep=0.1cm
\parsep=0.1cm
\itemsep=0.0cm
\begin{document}
\title{
Schelog\\
An embedding of Prolog in Scheme
}
\author{
(c) Dorai Sitaram, dorai@cs.rice.edu, 1989, Rice University
}
\date{Revised Feb. 1993
}
\maketitle

\medskip
{\baselineskip=10pt
\begin{verbatim}
Installing Schelog
******************
\end{verbatim}}
\medskip
\par
To use Schelog, the Prolog-in-Scheme provided here, you
may have to pre-load some features not already present
in your dialect.  The details and several Scheme files
for accomplishing the task are given in the file \verb+readme+
in the subdirectory \verb+portable/+ of this distribution.
Use the appropriate files therefrom.
\medskip
\par
Once you've loaded the init file, load the file
\verb+schelog.scm+.  This gives you the Prolog embedding.
There is no need to type anything to go from Scheme to
Prolog --- you can use both simultaneously to the extent
of intertwining both languages in the same
s-expression.  The best of both worlds thing.
\medskip
\par
The other Scheme files in this distribution are example
files, most of the examples coming from Sterling and
Shapiro's book, "The Art of Prolog".  These should give a
fair idea as to how to write and run Prolog programs in
Schelog.
\medskip
\par
It is beyond the scope of this documentation to tell you
\_how\_ the embedding works.  The literature abounds in
excellent treatments of implementations of Prolog in lisp
and Scheme.  This implementation is an embedding of the type
described by Christopher Haynes in "Logic continuations", J.
Logic Program. 4, 1987, p. 157-176 and by Matthias Felleisen
in "Transliterating Prolog into Scheme", tech report \#182,
Indiana Univ. Comp. Sci. Dept., 1985.  I.e., it uses
Scheme's first-class full continuations to implement
Prolog's backtracking mechanism.
\medskip
\newpage
{\baselineskip=10pt
\begin{verbatim}
Using Schelog
*************
\end{verbatim}}
\medskip
\par
The syntax of the Prolog provided here may be unusual.
\medskip
\par
This note assumes you already know enough of Prolog and
Scheme, at least syntaxwise.  I'll show a few examples
whereby you can figure out how to transfer your knowledge of
writing programs in "real" Prolog to writing in Schelog.
\medskip
\par
E.g., the member predicate in "real" Prolog reads
\medskip
{\baselineskip=10pt
\begin{verbatim}
        member(X, [X|Xs]).
        member(X, [Y|Ys]) :- member(X, Ys).
\end{verbatim}}
\medskip
\par
The same program in Schelog reads
\medskip
{\baselineskip=10pt
\begin{verbatim}
        (define %member
          (rel (x xs y ys)
            ((x (cons x xs)))
            ((x (cons y ys)) (%member x ys))))
\end{verbatim}}
\medskip
\par
It is a convention --- which can be flouted --- that all
Schelog predicate names start with \verb+%+.  This is solely to
avoid confusion with Scheme procedures of the same name.
Thus, the Schelog predicate for \verb+append+ is named \verb+%append+ so
as not to clash with Scheme's \verb+append+.  The \verb+%+ is not
mandatory however --- you may use any naming system that
pleases you.
\medskip
\par
Relations are defined using the form \verb+rel+.  \verb+rel+ is followed
by a list of identifiers, these being the names of the logic
variables used in the definition of the relation.  There is
no naming convention involved here such as the
initial-capital convention of Prolog.  \verb+rel+ introduces the
list of rules corresponding to the relation.
\medskip
\par
For Prolog's \verb+|+, simply use Scheme's cons.  In general,
Scheme's data structures can be used without any massaging
in Schelog.
\medskip
\par
The if-then-else predicate in "real" Prolog reads
\medskip
{\baselineskip=10pt
\begin{verbatim}
        if_then_else(P, Q, R) :- P, !, Q.
        if_then_else(P, Q, R) :- R.
\end{verbatim}}
\medskip
\par
The same thing in Schelog reads
\medskip
{\baselineskip=10pt
\begin{verbatim}
        (define %if-then-else
          (rel (p q r)
            ((p q r) p ! q)
            ((p q r) r)))
\end{verbatim}}
\medskip
\par
The cut is written \verb+!+, as in Prolog.  Anonymous variables are
written \verb+(_)+ --- thus \verb+%member+ could be rewritten as
\medskip
{\baselineskip=10pt
\begin{verbatim}
        (define %member
          (rel (x xs)
            ((x (cons x (_))))
            ((x (cons (_) xs)) (%member x xs))))
\end{verbatim}}
\medskip
\par
which corresponds to
\medskip
{\baselineskip=10pt
\begin{verbatim}
        member(X, [X|_]).
        member(X, [_|Xs]) :- member(X, Xs).
\end{verbatim}}
\medskip
\par
This should give you a feel for Schelog's syntax.  If not,
read some of the example files provided.
\medskip
\par
The interactive Prolog queries (?-) are handled with the
form `\verb+which+'.  Type a which-query just as you would any
Scheme expression that you'd want to evaluate, i.e., at the
Scheme prompt.
\medskip
{\baselineskip=10pt
\begin{verbatim}
        > (which () (%member 1 '(1 2 3)))
\end{verbatim}}
\medskip
\par
corresponds to
\medskip
{\baselineskip=10pt
\begin{verbatim}
        ?- member(1, [1, 2, 3])
\end{verbatim}}
\medskip
\par
and returns
\medskip
{\baselineskip=10pt
\begin{verbatim}
        ()
\end{verbatim}}
\medskip
\par
This means that the goal succeeded, but since no variables
were requested in the answer, you get an empty list.  To get
more solutions, type \verb+(more)+.  This is like saying yes to
Prolog's more? prompt.  Here, for instance, typing \verb+(more)+
gives you \verb+#f+, falsity signifying that there are no alternate
solutions to this goal.  (Note that this distinction between
a false answer and a true answer with no variables to
instantiate is lost in Schemes where \verb+#f+ and \verb+()+ are
identical.  Mercifully, this is easily remedied --- simply
use a dummy which-variable in such cases.)
\medskip
\par
For another example, consider the query
\medskip
{\baselineskip=10pt
\begin{verbatim}
        > (which (x) (%member x '(1 2 3)))
\end{verbatim}}
\medskip
\par
Here you want an instantiation for \verb+x+ in the solution.  Sure
enough, the result of this expression is
\medskip
{\baselineskip=10pt
\begin{verbatim}
        ((x 1))
\end{verbatim}}
\medskip
\par
viz., a list containing the logic variable bindings
requested.  Here there is only one variable \verb+x+, and its
binding is \verb+1+.
\medskip
\par
Typing \verb+(more)+ gives more solutions
\medskip
{\baselineskip=10pt
\begin{verbatim}
        > (more)
        ((x 2))
        > (more)
        ((x 3))
        > (more)
        #f
\end{verbatim}}
\medskip
\par
The final \verb+#f+ shows that there are no more solutions.
\medskip
\par
One could also have queries of the form
\medskip
{\baselineskip=10pt
\begin{verbatim}
        > (letref (x ...) (which (y ...) query))
\end{verbatim}}
\medskip
\par
Both \verb+letref+ and \verb+which+ introduce local logic variables (much
like Scheme's \verb+let+).  However, in the solutions, only the
which-variables are enumerated.  E.g.,
\medskip
{\baselineskip=10pt
\begin{verbatim}
        > (letref (x) (which () (%member x '(1 2 3))))
\end{verbatim}}
\medskip
\par
succeeds three times, without giving the values of \verb+x+.
\medskip
{\baselineskip=10pt
\begin{verbatim}
        ()
        > (more)
        ()
        > (more)
        ()
        > (more)
        ()
        > (more)
        #f
\end{verbatim}}
\medskip
\par
Since Schelog relations are just Scheme procedures, one can
use lexical scoping to define auxiliary relations, e.g.,
\verb+%reverse+ using an auxiliary that employs an accumulator (see
\verb+toys.scm+).  One is also not tied to the Prolog style ---
regular Scheme can be used too, treating Prolog relations as
just another paradigm used for local convenience along with
the other paradigms of Scheme.
\end{document}
